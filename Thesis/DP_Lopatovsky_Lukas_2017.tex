% options:
% thesis=B bachelor's thesis
% thesis=M master's thesis
% czech thesis in Czech language
% english thesis in English language
% hidelinks remove colour boxes around hyperlinks

\documentclass[thesis=M,english]{FITthesis}[2012/10/20]

% \usepackage[utf8]{inputenc} % LaTeX source encoded as UTF-8
% \usepackage[latin2]{inputenc} % LaTeX source encoded as ISO-8859-2
% \usepackage[cp1250]{inputenc} % LaTeX source encoded as Windows-1250

\usepackage{graphicx} %graphics files inclusion
\usepackage{graphicx}
\usepackage{subfig}
\usepackage{amsmath}
\usepackage{amssymb}
\usepackage[chapter]{algorithm} % http://ctan.org/pkg/algorithms
\usepackage[noend]{algpseudocode} % http://ctan.org/pkg/algorithmicx
\usepackage{pseudocode}
\usepackage{comment}
\usepackage{enumitem}
\usepackage{amssymb}

\usepackage{dirtree} %directory tree visualisation

% % list of acronyms
% \usepackage[acronym,nonumberlist,toc,numberedsection=autolabel]{glossaries}
% \iflanguage{czech}{\renewcommand*{\acronymname}{Seznam pou{\v z}it{\' y}ch zkratek}}{}
% \makeglossaries

%%%%%%%%%%%% own commands %%%%%%%%%%%%%%%%%%
\newcommand{\matr}[1]{\mathbf{#1}} 
\newcommand{\argmax}{\mathop{\mathrm{argmax}}}
%\newcommand{\max}{\mathop{\mathrm{max}}}

% % % % % % % % % % % % % % % % % % % % % % % % % % % % % % 
% EDIT THIS
% % % % % % % % % % % % % % % % % % % % % % % % % % % % % % 

\department{Department of Theoretical Computer Science }
\title{Thesis title (SPECIFY)}
\authorGN{Luk{\' a}{\v s}} %author's given name/names
\authorFN{Lopatovsk{\' y}} %author's surname
\author{Luk{\' a}{\v s} Lopatovsk{\' y}} %author's name without academic degrees
\authorWithDegrees{Bc. Luk{\' a}{\v s} Lopatovsk{\' y}} %author's name with academic degrees
\supervisor{Ing. Daniel Va{v s}ata, Ph.D.}
\acknowledgements{THANKS (remove entirely in case you do not with to thank anyone)}
\abstractEN{Summarize the contents and contribution of your work in a few sentences in English language.}
\abstractCS{V n{\v e}kolika v{\v e}t{\' a}ch shr{\v n}te obsah a p{\v r}{\' i}nos t{\' e}to pr{\' a}ce v {\v c}esk{\' e}m jazyce.}
\placeForDeclarationOfAuthenticity{Prague}
\keywordsCS{Replace with comma-separated list of keywords in Czech.}
\keywordsEN{HMM, CT-HMM.}
\declarationOfAuthenticityOption{1} %select as appropriate, according to the desired license (integer 1-6)
% \website{http://site.example/thesis} %optional thesis URL


\begin{document}

% \newacronym{CVUT}{{\v C}VUT}{{\v C}esk{\' e} vysok{\' e} u{\v c}en{\' i} technick{\' e} v Praze}
% \newacronym{FIT}{FIT}{Fakulta informa{\v c}n{\' i}ch technologi{\' i}}

\setsecnumdepth{part}

\begin{introduction}
	This is the master thesis. welcome!
	\section{Motivation and objectives}
	About a discrete model [inspired by Rabiner? ] and why it is not satisfactory  %(as http://delivery.acm.org/10.1145/350000/343402/p162-aziz.pdf?ip=147.32.98.33&id=343402&acc=ACTIVE%20SERVICE&key=D6C3EEB3AD96C931%2E9BD1EC80ACA8C1C5%2E4D4702B0C3E38B35%2E4D4702B0C3E38B35&CFID=702042173&CFTOKEN=62280276&__acm__=1481276923_1a71eb623e1c42d4c8e9b84178eb248f)	
	Add articles and DP that have done it before - less or more successfully 
\end{introduction}

\setsecnumdepth{all}

%TODO change chapter order and section affiliation.

\chapter{Expectation-Maximization Algorithm}\label{ch:EM}


%(1) Maximum Likelihood from Incomplete Data via the EM Algorithm , A. P. Dempster; N. M. Laird; D. B. Rubin, Journal of the Royal Statistical Society. Series B (Methodological), Vol. 39, No. 1. (1977), pp.  1-38.

Expectation-Maximization (EM) Algorithm (first introduces in [1]) is the method for finding the Maximum likelihood estimates(MLE) of parameters in probabilistic models over the incomplete data-set ( i.e. data-set containing unobserved (latent) variables ). It is a natural generalization of maximum likeli-
hood estimation, however the latent variables makes finding of the MLE more difficult. To count it directly can be very computationally expensive, so the EM algorithm use the iterative approach to approximate the solution by repeating the sequence of simpler consecutive steps.

Let's have vector $\mathbf{x} = (x_{1},x_{2},\dotsc,x_{n})$ containing the known (observed) variables and vector $\mathbf{z} = (z_{1},z_{2},\dotsc,z_{n})$ containing latent (unobserved) variables. We will set $\theta$ for the unknown parameters we want to estimate. We will use the notation $\mathrm{logP}(\mathbf{x},\mathbf{z};\theta)$ as the log-likelihood function, which estimates the probabilities for the given parameters. Notice that we are using the log-likelihood to avoid underflow of parameters values that could otherwise easily arise during the computation.  
%[st]simle tutorial: What is the expectation maximization algorithm? Chuong B Do & Serafim Batzoglou

Look at [st] for more detailed, easily understandable tutorial with examples.    

\begin{enumerate}
\item \textbf{Initialization:} We will make the initial guess of the parameters $\hat \theta^{(0)} $.
\end{enumerate} 

We will continue by iterating through two following steps: 

\begin{enumerate}[resume]
\item \textbf{Expectation:} Calculate the expected value of the likelihood function $\mathrm{logP}(\mathbf{x},\mathbf{z};\hat \theta^{(t)})$. Where $\hat \theta^{(0)}$ is the current parameters estimation. We will find the function $\mathrm{Q}$ that lower bounds $\mathrm{logP}(\mathbf{x},\mathbf{z};\hat \theta^{(t)})$.

\begin{equation}\label{eq:exp}
 \mathrm{Q}(\hat\theta^{(t)}) =  \mathrm{logP}(\mathbf{x},\mathbf{z};\hat \theta^{(t)})
\end{equation}

\item \textbf{Maximization:} Find the value $\hat\theta^{(t+1)}$ that maximize $\mathrm{Q}(\hat\theta^{(t)})$.  
\end{enumerate}  

As the value of $\mathrm{Q}(\hat\theta^{(t)})$ matches the log-likelihood function at $\hat\theta^{(t)}$, it follows that $ \mathrm{Q}(\hat\theta^{(t)}) =  \mathrm{logP}(\mathbf{x},\mathbf{z};\hat \theta^{(t)}) \leq \mathrm{Q}(\hat\theta^{(t+1)}) =  \mathrm{logP}(\mathbf{x},\mathbf{z};\hat \theta^{(t+1)}) $ thus the function value form the non-decreasing order. [nejaky dokaz?]. We will stop the computation ones the parameter converges to some value, that is not changing.  

It is important to notice that this approach leads to the local optimum. To get better results the algorithm can be launched more times starting always with different random initialization. So much some other optimization technique can be used [sources?]. 



\chapter{Discrete-time Hidden Markov Model}
This chapter will briefly discuss Discrete-time Hidden Markov Model (later just HMM) as defined in [Rabiner], where you can also look for the more comprehensive explanation with examples.

We have decided to write this chapter, because Continuous-time HMM is its direct logical extension and also it is using some of the subroutines that are the same as if defined for discrete time model.

\section{Discrete Markov Process}\label{sec:DMP}
    
Discrete Markov Process is the stochastic process $X_t$ in discrete equidistant time $\matr{t} = \{ t_1,t_2,\cdots \}$, that in every time occupies some state from the finite state set $\mathcal{S} = \{ s_1, s_2, \cdots, s_n\}$. We will denote as $X_i = s_j$, if the process in time $t$ is in the state $s_j$. The state transition probability will be given in matrix $\matr{A}$, where its elements $a_{ij}, 1\leq i,j \leq n$ represents the probability of the process to move from the state $s_i$ to the $s_j$ in any time step $t$.  Note that $i$ can equal $j$, in that case the process will not move, but it  will remain in the same state. The probability of it is $a_{ii}$.

To tell about the system to be Markov, it should hold so called Markov property. It can be expressed by equation \eqref{eq:mp}. It means that the probability of transition to another state depends only on the current state and doesn't use the information about any past states - the system is memory-less.

\begin{equation}\label{eq:mp}
\begin{aligned}
\mathrm{P}(X_t & = s_j \mid X_{t-1} = s_{j-1}, X_{t-2} = s_{j-2}, \cdots)  \\   
               & = \mathrm{P}(X_t = s_j \mid X_{t-1} = s_{j-1} )
\end{aligned}
\end{equation}

The other important property is Homogeneity. The probability of transition from one state to another is constant throughout the time. 

\begin{equation}\label{eq:homo}
   a_{ij} = \mathrm{P}(X_t = s_j \mid X_{t-1} = s_{j-1} ),\qquad 1 \leq i,j \leq n, \forall t
\end{equation}

For the transition probabilities are applied classical stochastic constrains.

\begin{equation}
   a_{ij} \geq 0
\end{equation}

\begin{equation}
   \sum_{j=1}^n a_{ij} = 1
\end{equation}

\section{Discrete-time Hidden Markov Model}

Discrete-time Hidden Markov Model (HMM) is the extension of the previously defined model~\ref{sec:DMP}. In the classic Markov process are the states directly observable. Such a model is often too simplistic to fit well real life problems. The HMM forms the doubly embedded stochastic process, in which the states are not directly observable, but only through the another dependable process, that to every state $s_i$ assign the probability, that it will emit the variable from the set $\mathcal{V}=\{  v_1,v_2,\cdots,v_m\}$. The $\mathcal{V}$ is the set of observable variables. It forms the part of the model, which is known and can be used to guess the possible underlying states chain. The probability of emission of variable $o_j$ by the state $s_i$ will be denoted as $b_i(j)$ and form the matrix $B$, the rectangular matrix of $n$ rows and $m$ columns.
  
To summarize the HMM can be described by its:

\begin{enumerate}[resume]
\setcounter{enumi}{0}
\item \textbf{Hidden States}
\begin{equation}
\mathcal{S} = \{ s_1,s_2, \cdots, s_n \}
\end{equation} 
\item \textbf{State Transition Probability Distribution}
\begin{equation}\label{eq:tp}
\matr{A} = \{ a_{ij} \}, \quad 1 \leq i,j \leq n
\end{equation} 
\item \textbf{Observation Symbols}
\begin{equation}
\mathcal{V} = \{ v_1,v_2, \cdots, v_m \} 
\end{equation}
\item \textbf{Observation Symbols Probability Distribution} \\
$\matr{B} = \{ b_{i}(k) \}$, where $b_{i}(k)$ is the probability that the observation $k$ will occur, if the system is currently in state $i$. 
\begin{equation}
b_i(k) = \mathrm{P}(v_k \text{ at t } \mid X_t = s_i), \qquad 1 \leq i \leq n, \quad 1 \leq k \leq m
\end{equation}
\item \textbf{Initial state distribution} \\
$\pi = \{ \pi_i \}$, where $\pi_i$ is the probability of the initial state being $S_i$.
\begin{equation}
\pi_{i} = \mathrm{P}(X_1 = s_i), \qquad 1 \leq i \leq n
\end{equation}
\end{enumerate}

For the convenience we will declare parameter $\lambda = \{\matr{A},\matr{B},\pi\}$, compactly denoting the set of all parameters of the model.   

\section{Three Basic Problems for HMMs}\label{sec:3p}
% citate: Jack Ferguson of IDA in lectures at Bell laboratories
When dealing with real-world application we need to deal with following problems, as proposed in [citace]. First we will describe the problems and later in the successive subsections we will explain the algorithms that can effectively solve them. 

\begin{enumerate}
\item Compute the probability $\mathrm{P}(\matr{O}|\lambda) $ of the observation sequence \\ $\matr{O} = (o_1,o_2,\cdots,o_T)$, given the set of parameters $\lambda = \{A,B,\pi\}$. Elements of the observation sequence $\matr{O}$ are some specific measured data, taking values from the set~$\mathcal{V}$.   
\item Choose the optimal state sequence $\matr{X} = (x_1,x_2,\cdots,x_T$), having the observation sequence $\matr{O}$ and parameters $\lambda$.
\item Adjust the model parameters $\lambda$ in the way it maximizes the probability of observation sequence $ \mathrm{P}(\matr{O}|\lambda) $. 
\end{enumerate}


\subsection{Forward-Backward Algorithm}
The Forward-Backward Algorithm is actually the pair of two separate algorithms (Forward vs. Backward). We will explain the Forward one and at the end we will describe the modifications that are needed to do the Backward. Both of them are sufficient for solving of the first proposed problem, however we need to define both of them for the later use in the text.  

Forward Algorithm is the Dynamic Programming algorithm building upon the Markov property - the independence upon past events. We will denote the forward variable as $\alpha_t(i)$ defined as the probability of partial observation sequence, with the last observation in time $t$ emitted by the state $s_i$, all given the parameters $\lambda$.

\begin{equation}
\alpha_t(i) = \mathrm{P}(o_1,o_2,\cdots,o_t,X_t = s_i \mid \lambda )
\end{equation}

that can be gradually counted by following equations for $t=1$ and $t=t+1$ using bottom-up strategy.

\begin{equation}
\alpha_1(i) = \pi_i b_i(o_1), \qquad 1 \leq i \leq n
\end{equation}

\begin{equation}
\begin{aligned}
\alpha_{t+1}(i) = \left( \sum_{j=1}^n \alpha_t(j) a_{ji} \right) b_i(o_{t+1}), \qquad 1& \leq t \leq T - 1, \\ 
                                                                                 1& \leq i \leq n		\end{aligned}
\end{equation}

Now we can obtain the solution of the first problem simply by summing through the all forward variables in the time $T$.

\begin{equation}
\mathrm{P}(\matr{O}|\lambda) = \sum_{i=1}^n \alpha_T(i)
\end{equation} 

Similarly we can define backward variable $\beta_t(i)$ as
\begin{equation}
\beta_t(i) = \mathrm{P}(o_{t+1},o_{t+2},\cdots,o_T,X_t = s_i \mid \lambda ) 
\end{equation}
The DP algorithm can be derived analogically as the one for the Forward procedure, with the single difference, that it will start counting from the end, so the total probability can be count by summing the backward variables in time $t=1$.

We need to count $n$ variables at each time-step, each of that takes exactly $n$ steps to evaluate. It makes overall complexity $\mathcal{O}(n^2T)$.    

\subsection{Individually Most Likely States Sequence}
There is more ways how we can look at the world "optimal" in the problem 2 statement. One of the possible approaches is to maximize the expected number of correctly assigned states. To solve it, we need to define the variable determining the probability of being in specific state in particular time.

\begin{equation}
\gamma_t(i) = \mathrm{P}(X_t = s_i \mid \matr{O},\lambda ) 
\end{equation}

Here we can use already defined forward and backward variables and count the $\gamma_t(i)$ as
\begin{equation}
\begin{aligned}
\gamma_t(i) &= \frac{ \mathrm{P}( X_t = s_i, \matr{O} \mid \lambda )}{ \mathrm{P}( \matr{O} \mid \lambda )} =
               \frac{  \alpha_t(i) \beta_t(i) }{ \mathrm{P}( \matr{O} \mid \lambda )} = \\
            &= \frac{  \alpha_t(i) \beta_t(i) }{ \sum\limits_{j=1}^n \alpha_t(j) \beta_t(j) } 
\end{aligned}
\end{equation}

To get the desired individual most likely state $x_t$, it is enough to find one of the highest probability.

\begin{equation}
x_t = \argmax_{1 \leq i \leq n} \gamma_t(i), \qquad 1 \leq t \leq T
\end{equation} 

Applying this algorithm to the whole sequence will lead to the highest expected number of correctly assigned states, however such a sequence as a whole can have low probability or in some extreme cases can not even be feasible. This would happen if probability of transition among two consecutive states in the sequence was zero.   
   
\subsection{Viterbi Algorithm} 

An another way how to look on the problem 2 is to find single most probable state sequence. It means to maximize $\mathrm{P}(\matr{X}\mid \matr{O},\lambda)$, what is equivalent to maximize $\mathrm{P}(\matr{X}, \matr{O}\mid\lambda)$. %TODO is \matr{X} defined?

Viterbi Algorithm is the Dynamic Programming algorithm, that similarly to the Forward-Backward algorithm  benefits from memorylessness of Markov Chain. In DP we can gradually count the variable $\delta_t(i)$, that represents the maximal probability of the state chain from $time = 1$ till $t-1$, with the current state being $s_i$.

\begin{equation}\label{eq:delta}
\delta_t(i) = \max_{x_1,x_2,\cdots,x_{t-1}} \mathrm{P}( x_1,x_2,\cdots, x_t = s_i, o_1, o_2, \cdots, o_t \mid \lambda )
\end{equation} %TODO x_t = s_i, or X_t = s_i

To get the actual state sequence, we need to store the information about the state that has maximized in the previous equation \eqref{eq:delta}. We will store it in the array $\psi_t(i)$. Now we can define the initialization of the algorithm as

\begin{equation}
\delta_1(i) = \pi_i b_i(o_1), \qquad 1 \leq i \leq n 
\end{equation}

\begin{equation}
\psi_1(i) = 0 
\end{equation}

and consecutive bottom-up computation as

\begin{equation}
\begin{aligned}
\delta_{t}(i) = ( \max_{ 1 \leq j \leq n } \delta_{t-1}(j)a_{ji} ) b_i(o_{t}), \qquad 2& \leq t \leq T, \\
																					   1& \leq i \leq n
\end{aligned}
\end{equation}

\begin{equation}
\begin{aligned}
\psi_{t}(i) = ( \argmax_{ 1 \leq j \leq n } \delta_{t-1}(j)a_{ji} ), \qquad \qquad 2& \leq t \leq T, \\
																			       1& \leq i \leq n
\end{aligned}
\end{equation}

Now we can get the searched state sequence probability 

\begin{equation}
\mathrm{P}^* = \max_{1 \leq i \leq n} ( \delta_{t}(i) )  
\end{equation}

and the actual state path by backtracking

\begin{equation}
x_t^* = \argmax_{1 \leq i \leq n} ( \delta_{t}(i) ),  \qquad t = T  
\end{equation}

\begin{equation}
x_t^* = \psi_{t+1}(x_{t+1}^*), \qquad t = T-1, T-2, \cdots, 1  
\end{equation}

The structure of the algorithm is very similar to the Forward-Backward algorithm, so we can easily see its complexity is also $\mathcal{O}(n^2T)$.

\subsection{Baum-Welch Algorithm}\label{sec:BWA}

There doesn't exist analytic solution for the problem~3. There are more possible iterative algorithms from which we will describe the expectation-maximization approach~\ref{ch:EM}, based on the classic work of Baum and his colleagues.[ref?] We will start by defining the variable~$\xi_t(i,j)$ as the probability~of being in the state~$s_i$ in the time~$t$ and in state~$s_j$ in the time~$t+1$. 

\begin{equation}
\xi_t(i,j) = \mathrm{P}( X_t = s_i, X_{t+1} = s_j \mid \matr{O}, \lambda )  
\end{equation}

The probability can be compute using forward-backward variables as follows
 
\begin{equation}\label{eq:xi}
\begin{aligned}
\xi_t(i,j) &= \frac{ \alpha_t(i) a_{ij} b_j(o_{t+1}) \beta_{t+1}(j) }
		   		   { \mathrm{P}( \matr{O} \mid \lambda ) } \\
		   &= \frac{ \alpha_t(i) a_{ij} b_j(o_{t+1}) \beta_{t+1}(j) }
		   		   { \sum\limits_{i=1}^n \sum\limits_{j=1}^n \alpha_t(i) a_{ij} b_j(o_{t+1}) \beta_{t+1}(j) }
\end{aligned}
\end{equation}

Obviously it is in relation with already defined variable~$\gamma_t(i)$ - the probability of being in the state~$s_i$ in the time~$t$.

\begin{equation}
\gamma_t(i) = \sum_{j=1}^n \xi_t(i,j)  
\end{equation}

We can get the expected number of transitions from~$s_i$, when summing~$\gamma_t(i)$ over the time till~$T-1$. Similarly we would get expected number of times in~$s_i$, by summing over~$t$ from~$t=1$ till~$t=T$. 

\begin{equation}
\mathbf{E}(\text{transitions from $s_i$}) = \sum_{t=1}^{T-1} \gamma_t(i)  
\end{equation}

If summing~$\xi_t(i,j)$ over the time, we will get expected number of transitions from $s_i$ to $s_j$, . 

\begin{equation}
\mathbf{E}(\text{transitions from $s_i$ to $s_j$}) = \sum_{t=1}^{T-1} \xi_t(i,j)  
\end{equation}

Now we have all what is needed to define the reestimated model~$\hat\lambda=(\hat{\matr{A}},\hat{\matr{B}},\hat\pi)$.

\begin{equation}\label{eq:bwpi}
\hat\pi_i = \gamma_1(i)  
\end{equation}

\begin{equation}\label{eq:bwa}
\begin{aligned}
\hat a_{ij} &= \frac{\mathbf{E}(\text{transitions from $s_i$ to $s_j$})}
				   {\mathbf{E}(\text{transitions from $s_i$})}  \\
		    &= \frac{\sum\limits_{t=1}^{T-1} \xi_t(i,j)}{\sum\limits_{t=1}^{T-1} \gamma_t(i) }
\end{aligned}
\end{equation}

\begin{equation}\label{eq:bwb}
\begin{aligned}
\hat b_{i}(k) &= \frac{\mathbf{E}(\text{times of visiting $s_i$ and observing symbol $v_k$ })}
				   {\mathbf{E}(\text{times of visiting $s_i$})} \\
			  &= \frac{\sum\limits_{t=1, \text{if } O_t = v_k  }^{T} \gamma_t(i)}{\sum\limits_{t=1}^{T} \gamma_t(i) } 
\end{aligned}
\end{equation}

%TODO O_t = v_k, ok?

The formulas can be interpreted as the steps of EM-algorithm, with the expectation step being the computation of auxiliary function~$Q(\lambda,\hat\lambda)$ and maximization step, the maximization over parameter~$\hat\lambda$. The algorithm will stop ones it is no longer improving (re-estimation doesn't bring higher probability). It is important to notice, that the found solution is the local optimum, the searched space can be very complicated containing many local optima, that's way it can helpful to start the algorithm more times with different parameter initialization.  

%proof it is growing { was mentioned in EM }

\chapter{Continuous-Time Hidden Markov Model}

All the time we will mention continuous-time hidden Markov model, we will consider its finite states variant. Also the infinite states models exist, but they are out of the scope for these thesis.

In the DT-HMM described in previous chapter, the change of the current state of the process could happen once we moved a step further in the discrete time $t$. The actual state remained hidden, but it has always emitted the observable variable $v_i$.

Comparing to this, in the CT-HMM can change of the state occur at any moment in time (the occurrence holds exponential distribution ). Also it may not be documented by the observation. The emission of the observable variable can happen at any time, however it is independent on the state transition times. For example, it could be times, when the patient undergoes medical examination. The times can generally have highly irregular and unbalanced distribution.

There is much more latent information in such defined model. Not just the hidden states, but also the unknown transition times and unknown state sojourn time (how long will the system remain in the state).
Sometimes the state can change, without emitting a single observation. The large number of hidden information make it to be more complex problem then the discrete time model.

Now we will continue by formally defining Continuous time Markov process, after which we will extend it to the hidden model.

\section{Continuous-Time Markov Process} 

%relationship with discrete (construction)? sum(n = 0..inf) p(n(t)) u(i,j)^n 
%construction from discrete HMM is probably important, heterogenous MM.
%cit:[S.Karlin]
A finite state continuous time Markov process (later as CTMC) is a~stochastic process $X_t (t > 0)$ on the states $\mathcal{S} = \{ s_1, s_2, \cdots, s_n \}$  (for $n>0$ and $\mathcal{S}$ being the finite state set), that in any time $t$  occurs in the corresponding state $1 \leq x_t \leq n$. 

For the times $0 \leq u_0 < u_1 < \cdots < u_r < u$ it satisfies the following equations: 
\begin{itemize}
\item \textbf{Markov property:} Probability of transition from state $i$ to state $j$ during the time interval $t$ is stationary i.e. independent on the states of process in the times $< u$.  
    
\begin{equation}
\begin{aligned}
& \mathrm{P}( x_{t+u} = j \mid x_u = i, x_{u_r} = i_r, \cdots , x_{u_0} = i_0 ) = \\ 
& = \mathrm{P}( x_{t+u} = j \mid x_u = i ), \qquad 1 \leq i,j,i_{0,\cdots,r} \leq n 
\end{aligned}
\end{equation}

The equation describes that the stochastic process is memoryless.

\item \textbf{Homogeneity:} Probability of transition from the state $i$ to $j$ in any given time $s \geq 0$ depends only at the length of the time interval $t \geq 0$. 

\begin{equation}
\begin{aligned}
& \mathrm{P}( y_{t+s} = j \mid y_s = i ) = \mathrm{P}( y_t = j \mid y_0 = i ) = \\
& = \mathrm{p_t}(i,j)
\end{aligned}
\end{equation}

\end{itemize}
 

The upper mentioned conditions assert, that the transition probability $\mathrm{p_t}(i,j), 1 \leq i,j \leq S$ satisfies following conditions:%cit:[S.Karlin]


\begin{equation}
\mathrm{p_t}(i,j) \geq 0
\end{equation}

\begin{equation}
\sum_{j = 1}^n \mathrm{p_t}(i,j) = 1
\end{equation}

\begin{equation}\label{eq:l0}  
\lim_{t \to 0^+} \mathrm{p_t}(i,j)= 
\begin{cases}
1, i = j\\
0, i \neq j
\end{cases}
\end{equation}

\begin{equation}\label{eq:ckeq}
\sum_{k = 1}^n\mathrm{p_u}(i,k)\mathrm{p_t}(k,j) = \mathrm{p_{u+s}}(i,j)     
\end{equation}

Equation \eqref{eq:ckeq} is known as Chapman-Kolmogorov equation. 
We can define the matrix $\mathrm{P_t}$, where the entry $(i,j)$ is $\mathrm{p_t(i,j)}$ to get the equation in following form:

\begin{equation}\label{eq:ckm}
\matr{P_u} \matr{P_t} = \matr{P_{u+t}},   \qquad t,u > 0  
\end{equation}

?proof of Chapman-kolmogorov?

%construction from Discrete time with poisson timing

\subsection{ Jump Rates }

Using the Chapman-Kolmogorov equation \eqref{eq:ckeq}, if we know the probability $\mathrm{p_t(i,j)}$ for every states $i,j$ and time $0 < t < t_0$, we are able to compute the values for any time $t > 0$. 
%cit[vytistena st. kniha]

The property \eqref{eq:l0}  asserts that $\mathrm{p_t(i,j)}$ is continuous for $t=0$. It can be showed from equation \eqref{eq:ckm}, after assigning $t=0$ we will get identity matrix $\matr{P_0} = \matr{I}$.
Moreover from \eqref{eq:ckm} follows that $\mathrm{p_t(i,j)}$ is continuous for all $t>0$ ref[S.Karlin] and so that there exists the right derivative in 0. This knowledge enables us to determine the $\mathrm{p_t(i,j)}$ for any given time $t>0$.   

\begin{equation}
q_{ij} =  \left.\frac{\mathrm{d}\mathrm{p_t}(i,j)}{\mathrm{dt}} \,\right|_{t=0} = \lim_{t \to 0^+} \frac{\mathrm{p_t}(i,j)}{t}, \quad \text{if } i\neq j     
\end{equation}

We will call this derivative $q_{ij}$ the jump rate from state~$i$ to some other state~$j$. The jump rate $q_{ii}$ can be derived from the equation of transition probabilities summing to the one.

\begin{equation}
1 = \mathrm{p_t}(i,i) + \sum_{j = 1}^{ n ,j \neq i} \mathrm{p_t}(i,j) 
\end{equation}

dividing by time interval $t$ and letting $t$ decrease close to zero, we will obtain the following equation. 

\begin{equation}\label{eq:qii}
 q_{ii} =  \sum_{j = 1}^{ n ,j \neq i} q_{ij} 
\end{equation}

(We assume the finite rates $q_{ij}$, infinite rate would immediately lead to the leaving of the state, so it makes no sense for us to consider.)

It can be shown that by construction of the CTMC from discrete time Markov chain (DTMC) and  it's underlying Poisson process with rate $\lambda$, the equation $q_{ij} = \lambda a_{ij}$ holds. Where $a_{ij}$ is the transition probability from the $i$ to $j$ in DTMC \label{eq:tp}. This is why we call the $q_{ij}$ the jump rate.[zdroj]  

%ROUTING MATRIX - u(i, j) = q(i, j) / lambda max
%u(i, i) = 1 - sum_j u(i, j)

We can also look at $q_{ii}$ as at the rate in which the $X_t$ is leaving the state $i$ \eqref{eq:ckeq}. Then we can define the Infinitesimal matrix $\matr{Q}$ as following:

\begin{equation}
\begin{aligned}  
\matr{Q}(i,j)= 
\begin{cases}
\qquad q_{ij}, \qquad & \text{if } i\neq j\\
\sum\limits_{k = 1}^{ n ,k \neq i} q_{ik}, \qquad & \text{if } i=j
\end{cases}
\end{aligned}
\end{equation}

Such a defined matrix will hold $\pi\matr{Q} = 0$... [explain more, proof, $e^{\matr{Q}t}$]

\subsection{Construction from Discrete-Time Markov Process with Poisson Process Timing } 

The useful way to the understanding of the continuous-time Markov process is by its relation to the discrete process, that can be shown by the construction.

We will take the homogeneous Poisson process $N(t)$ with parameter $\lambda$ and the discrete-time Markov process $Y_n$, with the transition probabilities $a_{ij}$. With the $N(t)$ and $Y_n$ being mutually independent. To make the construction, we will change the equidistant timing of $Y_n$ by Poisson process timing. We will get the process $Y_{N(t)}$ in which the transition from the state $i$ happens at each arrival in $N(t)$ with the probability $a_i = \sum_{j = 1}^{ n ,j \neq i} a_{ij}$ (notice that the transition will not occur if $i=j$) and the sojourn time in state $i$ will be exponentially distributed with the parameter $q_i$, what is $\mathrm{f}(t)= a_i e^{-a_i t}$. Such a process $X_t = Y_{N(t)}$ is called continuous-time Markov process.

Now we can derive the equation for probability of transition from the state $i$ to the state $j$ in the time $t$ as the sum of all the possible number of steps in which the transition can occur. 

\begin{equation}
\begin{aligned}
\mathrm{p_t(i,j)} &= \sum_{n=0}^{\infty} \mathrm{P}( N(t) = n, Y_t = s_j \mid Y_t = s_i )  = \\
                  &= \sum_{n=0}^{\infty} \mathrm{P}( N(t) = n, Y_{N(t)} = s_j \mid Y_{N(t)} = s_i )  = \\
                  &= \sum_{n=0}^{\infty} \mathrm{P}( N(t) = n ) \mathrm{P}( Y_n = j \mid Y_0 = i )  = \\
				  &= \sum_{n=0}^{\infty} e^{-\lambda t} \frac{ (\lambda t)^n}{n!} a_{ij}^n  
\end{aligned}
\end{equation} 


%ref Yu-Ying Liu

\subsection{ Fully Observable Continuous-Time Markov Process }

Let's have continuous time Markov chain on the state space $S$, with the jump rates in infinitesimal matrix $Q$, initial state probability distribution $\pi$ and the known transitions to the corresponding states $Y^{'}= \{y_0, y_1, \cdots, y_{V^{'}} \} $ occurring at times $T^{'} = \{ t_0^{'}, t_1^{'}, \cdots, t_{V^{'}}^{'} \}$. Such a system, where we know, when and which the transition will happen is called fully observable and we can count its complete likelihood ($\mathcal{CL}$). 

\begin{equation}\label{eq:CL1}
\begin{aligned}  
 \mathcal{CL} &=  \prod_{v^{'}}^{V^{'}-1} ( q_{y_{v^{'}} y_{v^{'}+1}} / q_{y_{v^{'}}} )( q_{y_{v^{'}}} e^{ - q_{y_{v^{'}}} \tau_{v^{'}} }) = \\
    &= \prod_{v^{'}}^{V^{'}-1} q_{y_{v^{'}} y_{v^{'}+1}} e^{ - q_{y_{v^{'}}} \tau_{v^{'}} }
\end{aligned}
\end{equation}

where the $q_i = \sum_{j\in S, i\neq j} q_{ij}$ is the probability of transition from the state~$i$ and   
$\tau_{v^{'}} = t_{v^{'}+1}^{'} - t_{v^{'}}^{'}$ is the time interval among the two consecutive steps.

The equation can be further rearranged in the form that group together the same state transition. The variable $n_{ij}$ marks the number of transition $q_ij$ that have occurred and the~$\tau_i$ is the total time spend in the state~$i$.

\begin{equation}\label{eq:CL2}
 \mathcal{CL} = \prod_{i \in S} \prod_{j \in S, i \neq j} q_{ij}^{n_{ij} } e^{ - q_i \tau_i }
\end{equation}

\subsection{ General Continuous-Time Markov Process }

In general we do not know the state of the system during all the time. It is only known at some unevenly distributed times $T = \{ t_0, t_1, \cdots, t_{V} \}$ as $Y= \{y_0, y_1, \cdots, y_{V} \}$. 
This add an amount of insecurity in the probability computation. We do not longer know, the number of transitions $n_{ij}$ as well as the time spend in the specific state $\tau_i$. 

To count the likelihood of the chain, we will use earlier defined matrix $\matr{P}(t)$ [ref] for the time interval~$\tau_v = t_{v+1} - t_v$.    

\begin{equation}
 \mathcal{L} = \prod_{v=0}^{V-1} \matr{P}_{y_v y_{v+1}}(\tau_v) 
\end{equation}

It can be alternatively extended in the form

\begin{equation}
 \mathcal{L} = \prod_{v=0}^{V-1} \prod_{i,j \in S}  \matr{P}_{ij}(\tau_v)^{\mathbb{I}( y_v = i, y_{v+1} = j )} 
\end{equation}

where the function $\mathbb{I}(i,j)$ equal either $1$, if both condition inside are true, or $0$ if they are not. 
If the variable $r$ - the number of all distinct time intervals $\tau_{\Delta}, \Delta =\{1,2,\cdots,r\}$ is lower then number of observations, it could be beneficial to aggregate them as in formula,

\begin{equation}\label{eq:CTL}
 \mathcal{L} = \prod_{\Delta = 1}^{r} \prod_{i,j \in S}  \matr{P}_{ij}(\tau_{\Delta})^{\mathbb{C}( \tau=\tau_{\Delta} y_v = i, y_{v+1} = j )} 
\end{equation}

where function $\mathbb{C}$ denotes the total number of intervals, for that condition is true.
    
%TODO P_{ij}(t) counting possibilities.    
    
\subsection{EM Algorithm} 

There is no available analytic maximizer of likelihood function \eqref{eq:CTL}, however it can be estimated iteratively by the EM algorithm proposed in [ref:PhysRevE.76].  
For the expectation step we will use logarithms of expected complete likelihood function \eqref{eq:CL2}, with the estimated parameter of $Q$ in step $t$ noted as $\hat Q_t $.
%TODO Q is not bold here \matr

\begin{equation}\label{eq:EMCL}
 \ln( \mathcal{CL} ) = \sum_{i \in S} \sum_{j \in S, i \neq j} \ln( \hat q_{ij}) \mathbf{E}( n_{ij} \mid Y, \hat Q_t ) - \hat q_i \mathbf{E}( \tau_i \mid Y, \hat Q_t )
\end{equation}

In the maximization step we will evaluate new values for $\hat Q$, so that the previous equation was maximized. 

\begin{equation}
\begin{aligned}  
\hat Q_t(i,j)= 
\begin{cases}
\frac{ \mathbf{E}(n_{ij} \mid Y, \hat Q_t )}{ \mathbf{E}( \tau_i \mid Y, \hat Q_t ) } & \text{if } i\neq j\\
- \sum\limits_{k \in S ,k \neq i} q_{ik}, \qquad & \text{if } i=j
\end{cases}
\end{aligned}
\end{equation}

Now, there remain the non-trivial task to evaluate $\mathbf{E}( n_{ij} \mid Y, \hat Q_t )$ and $\mathbf{E}( \tau_i \mid Y, \hat Q_t )$. They can be expressed as following sums.


\begin{equation}
\begin{aligned}  
\mathbf{E}(n_{ij} \mid Y, \hat Q_t ) &= \sum_{v=0}^{V-1} \mathbf{E}(n_{ij} \mid y_v, y_{v+1}, \hat Q_t ) = \\
&= \sum_{v=0}^{V-1} \sum_{k,l \in S} \mathbb{I}( y_v = k, y_{v+1} = l ) \mathbf{E}(n_{ij} \mid y_v = k, y_{v+1} = l, \hat Q_t )
\end{aligned}
\end{equation}


\begin{equation}
\begin{aligned}  
\mathbf{E}( \tau_i \mid Y, \hat Q_t ) &= \sum_{v=0}^{V-1} \mathbf{E}(\tau_i \mid y_v, y_{v+1}, \hat Q_t ) = \\
&= \sum_{v=0}^{V-1} \sum_{k,l \in S} \mathbb{I}( y_v = k, y_{v+1} = l ) \mathbf{E}(\tau_i \mid y_v = k, y_{v+1} = l, \hat Q_t )
\end{aligned}
\end{equation}
   
With the use of the Markov property and from homogeneity of the Markov process, the computations may be reduced to computing of expected values, for all foursome of $i,j,k,l \in S$. It will be showed later in the text[future ref].     
   
\section{Continuous-Time Hidden Markov Model}

Continuous-time hidden Markov model is the extension of CTMC, where the states in times $T = \{ t_0, t_1, \cdots, t_{V} \}$ are not directly observed, just seen as the observation symbols $O = \{  o_0, o_1, \cdots, o_{V} \}$, emitted by the current state $i$ with the probability $b_i(o)$.
The likelihood of the completely observed system  $\mathcal{CL}$ will be similar to the $\mathcal{CL}$ of CTMC \eqref{eq:CL1} \eqref{eq:CL2} with the difference we need to take into account the probability of actual observation.

\begin{equation}\label{eq:HMCL1}
\begin{aligned}  
 \mathcal{CL} &= \prod_{v^{'}}^{V^{'}-1} q_{y_{v^{'}} y_{v^{'}+1}} e^{ - q_{y_{v^{'}}} \tau_{v^{'}} } 
    \prod_{v=0}^V b_{s(t_v)}(o_v) = \\
    &= \prod_{i \in S} \prod_{j \in S, i \neq j} q_{ij}^{n_{ij} } e^{ - q_i \tau_i } \prod_{v=0}^V b_{s(t_v)}(o_v)
\end{aligned}
\end{equation}
%TODO s(t_v) was not defined enywhere, it need to be unified with the notation q_t = S from discrete model!

%TODO Move the CT-HMM intorduction in here, and separate it in the more chapters, not the EM- chapter.


\section{The Model Parameters Estimation}
%viterbi forward-backward need to be redefined with different rates matrix for different interval.

%continuous time Baum-Welch -> citace veci, ktore to robili - DP, spaniely, starsi clanok. + preco je novsi clanok najlepsi.

For the model parameters estimation in DT-HMM, we have used the already long known Baum-Welch algorithm \ref{sec:BWA}. We can use the same approach for the estimation of parameters $\hat \pi$ \eqref{eq:bwpi} and $\hat B$ \eqref{eq:bwb}. Just we need to use the time dependent jump rates instead of time-homogeneous transition probabilities. %TODO define new \gamma \chi ? maybe will be done in Posterior.  

However the estimation of the $\hat Q$ can't be inherited from the discrete model, because the state transitions are not dependent on the observation times and there can even be more of them between two observations. The following algorithm proposed in the article [ref:YuYingLiu], was the first which could efficiently solve the issue.        

The expectation step will be the log-likelihood of the model. Simply just the logarithm of the full observation probability function \eqref{eq:HMCL1}.

\begin{equation}\label{eq:EMCTHMM}
\begin{aligned}  
 \ln(\mathcal{CL}) &= \sum_{i \in S} \sum_{j \in S, i \neq j} \big( \ln( \hat q_{ij}) \mathbf{E}( n_{ij} \mid O,T, \hat Q_t ) - \hat q_i \mathbf{E}( \tau_i \mid O,T, \hat Q_t ) \big) + \\
    &+ \sum_{v=0}^V \mathbf{E}( \ln( b_{s(t_v)}(o_v) ) \mid, O,T,\hat Q_t )
\end{aligned}
\end{equation}

We will maximize the previous function by recalculating the~$\hat Q_t$. 

\begin{equation}
\begin{aligned}  
\hat Q_t(i,j)= 
\begin{cases}
\frac{ \mathbf{E}(n_{ij} \mid O,T, \hat Q_t )}{ \mathbf{E}( \tau_i \mid O,T, \hat Q_t ) } & \text{if } i\neq j\\
- \sum\limits_{k \in S ,k \neq i} q_{ik}, \qquad & \text{if } i=j
\end{cases}
\end{aligned}
\end{equation}

The challenging step is to derive computation of expected values of $n_{ij}$ and $\tau_i$. We will start with the first $\mathbf{E}(n_{ij} \mid O,T, \hat Q_t )$, which will be expressed as the sum through all possible state sequences.

\begin{equation} 
\sum_{s(t_1),\cdots,s(t_V)} p(s(t_1),\cdots,s(t_V) \mid O,T, \hat Q_t) \mathbf{E}(n_{ij} \mid s(t_1),\cdots,s(t_V) ,\hat Q_t )
\end{equation}

The expectation part can be rewrite in the intervals form.  

\begin{equation}
 \sum_{s(t_1),\cdots,s(t_V)} p(s(t_1),\cdots,s(t_V) \mid O,T, \hat Q_t) \sum_{v=1}^{V-1}\mathbf{E}(n_{ij} \mid s(t_v),s(t_{v+1}) ,\hat Q_t )
\end{equation}

Because the Markov condition hold, we can interval-wise divide the complex probability function and for every of the time intervals sum through all possible states $k,l$ at its edge. 

\begin{equation}
\sum_{v=0}^{V-1} \sum_{k,l \in S} p(s(t_v) = k,s(t_{v+1}) = l \mid O,T, \hat Q_t) \mathbf{E}(n_{ij} \mid s(t_v) = k,s(t_{v+1}) = l,\hat Q_t )
\end{equation}

The expected value of sojourn time $\mathbf{E}( \tau_i \mid O,T, \hat Q_t )$ can be derived equivalently.

\begin{equation}
\sum_{v=0}^{V-1} \sum_{k,l \in S} p(s(t_v) = k,s(t_{v+1}) = l \mid O,T, \hat Q_t) \mathbf{E}(\tau_i \mid s(t_v) = k,s(t_{v+1}) = l,\hat Q_t )
\end{equation}

In the following section \ref{sec:pos}, we will show, how to count $p(s(t_v) = k,s(t_{v+1}) = l \mid O,T, \hat Q_t)$. The remaining part of computing $\mathbf{E}(n_{ij} \mid s(t_v) = k,s(t_{v+1}) = l,\hat Q_t )$ and $\mathbf{E}(\tau_i \mid s(t_v) = k,s(t_{v+1}) = l,\hat Q_t )$ will be presented in [future ref]. 

\section{The Posterior State Probabilities}\label{sec:pos}

We have introduced the tree basic problems of DT-HMM in section \ref{sec:3p}. They can be likewise defined for the continuous model. We have already showed the partial solution of the third problem (model parameters adjustment) in the previous section, now we will discuss the first two problems. We will show that to solve them, it is needed only little transformation of DT-HMM algorithms.

To count the probability of the CT-HMM with specified parameters, and defined states in observation points, we do not need to know about the state transition that have occurred among the observations.
When we take this observation into the account, we can look on the model as it was the time in-homogeneous DT-HMM. It is the model, in which the state transition probabilities can change in every time-step. We have previously defined the transition probability from state $i$ to the state $j$ in time interval $\tau$ as $p_{\tau}(i,j) = e^{Q \tau}$, that can play the role of transition probabilities in discrete time in-homogeneous model. Now we can define the desired probability as following.

\begin{equation}
\prod_{v=1}^{V-1} p_{\tau_v}(s(t_v),s(t_{v+1})) \prod_{v=1}^{V} b_{s(t_v)}(o_v)
\end{equation}

It should obvious that for the continuous variant of Viterbi and Forward-backward algorithm, it is only needed to change the state transition $a_{ij}$ to time variable~$p_{\tau_v}(i,j)$.   
Now, we have all what we need to for counting of posterior state distribution $p(s(t_v) = k,s(t_{v+1}) = l \mid O,T, \hat Q_t)$, using the forward and backward variables as we hove done, while defining the variable~$\xi_t(i,j)$ in the discrete model \eqref{eq:xi}.

The described way for computing the posterior distribution of hidden states is referred to as the "Soft"~method. Supplementary, there is the "Hard"~method, that use the Viterbi algorithm to get the maximal posterior assignment of the states. ??todo (1-0) ??  The Hard method is the approximation, but can spare some computation time. ??how much, is it reasonable??           
    



%\chapter{EM Algorithms for CT-HMM}



%\chapter{Realisation}

\setsecnumdepth{part}
\chapter{Conclusion}


\bibliographystyle{iso690}
\bibliography{mybibliographyfile}

\setsecnumdepth{all}
\appendix

\chapter{Acronyms}
% \printglossaries
\begin{description}
	\item[DTMC] Discrete-time Markov Chain	
	\item[CTMC] Continuous-time Markov Chain
	\item[DT-HMM] Discrete-time Hidden Markov Model
	\item[CT-HMM] Continuous-time Hidden Markov Model
	\item[EM] Expectation-Maximization 
	\item[MLE] Maximum likelihood estimation
\end{description}


\chapter{Contents of enclosed CD}

%change appropriately

\begin{figure}
	\dirtree{%
		.1 readme.txt\DTcomment{the file with CD contents description}.
		.1 exe\DTcomment{the directory with executables}.
		.1 src\DTcomment{the directory of source codes}.
		.2 wbdcm\DTcomment{implementation sources}.
		.2 thesis\DTcomment{the directory of \LaTeX{} source codes of the thesis}.
		.1 text\DTcomment{the thesis text directory}.
		.2 thesis.pdf\DTcomment{the thesis text in PDF format}.
		.2 thesis.ps\DTcomment{the thesis text in PS format}.
	}
\end{figure}

\end{document}
