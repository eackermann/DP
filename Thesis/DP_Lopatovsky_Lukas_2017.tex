% options:
% thesis=B bachelor's thesis
% thesis=M master's thesis
% czech thesis in Czech language
% english thesis in English language
% hidelinks remove colour boxes around hyperlinks

\documentclass[thesis=M,english]{FITthesis}[2012/10/20]

% \usepackage[utf8]{inputenc} % LaTeX source encoded as UTF-8
% \usepackage[latin2]{inputenc} % LaTeX source encoded as ISO-8859-2
% \usepackage[cp1250]{inputenc} % LaTeX source encoded as Windows-1250

\usepackage{graphicx} %graphics files inclusion
\usepackage{graphicx}
\usepackage{subfig}
\usepackage{amsmath}
\usepackage{amssymb}
\usepackage[chapter]{algorithm} % http://ctan.org/pkg/algorithms
\usepackage[noend]{algpseudocode} % http://ctan.org/pkg/algorithmicx
\usepackage{pseudocode}
\usepackage{comment}



\usepackage{dirtree} %directory tree visualisation

% % list of acronyms
% \usepackage[acronym,nonumberlist,toc,numberedsection=autolabel]{glossaries}
% \iflanguage{czech}{\renewcommand*{\acronymname}{Seznam pou{\v z}it{\' y}ch zkratek}}{}
% \makeglossaries

% % % % % % % % % % % % % % % % % % % % % % % % % % % % % % 
% EDIT THIS
% % % % % % % % % % % % % % % % % % % % % % % % % % % % % % 

\department{Department of Theoretical Computer Science }
\title{Thesis title (SPECIFY)}
\authorGN{Luk{\' a}{\v s}} %author's given name/names
\authorFN{Lopatovsk{\' y}} %author's surname
\author{Luk{\' a}{\v s} Lopatovsk{\' y}} %author's name without academic degrees
\authorWithDegrees{Bc. Luk{\' a}{\v s} Lopatovsk{\' y}} %author's name with academic degrees
\supervisor{Ing. Daniel Vašata, Ph.D.}
\acknowledgements{THANKS (remove entirely in case you do not with to thank anyone)}
\abstractEN{Summarize the contents and contribution of your work in a few sentences in English language.}
\abstractCS{V n{\v e}kolika v{\v e}t{\' a}ch shr{\v n}te obsah a p{\v r}{\' i}nos t{\' e}to pr{\' a}ce v {\v c}esk{\' e}m jazyce.}
\placeForDeclarationOfAuthenticity{Prague}
\keywordsCS{Replace with comma-separated list of keywords in Czech.}
\keywordsEN{Fabulous, Great, Extraordinary.}
\declarationOfAuthenticityOption{1} %select as appropriate, according to the desired license (integer 1-6)
% \website{http://site.example/thesis} %optional thesis URL


\begin{document}

% \newacronym{CVUT}{{\v C}VUT}{{\v C}esk{\' e} vysok{\' e} u{\v c}en{\' i} technick{\' e} v Praze}
% \newacronym{FIT}{FIT}{Fakulta informa{\v c}n{\' i}ch technologi{\' i}}

\setsecnumdepth{part}

\begin{introduction}
	This is the master thesis. welcome!
	\section{Motivation and objectives}
\end{introduction}

\setsecnumdepth{all}
\chapter{Continuous-time Markov Chain}


\section{Definition}

\begin{equation}
\begin{aligned}
& \mathrm{P}( X_t+s = j \mid X_s = i, X_{s_n} = i_n, \cdots , X_{s_0} ) = \\ 
& = \mathrm{P}( X_t+s = j \mid X_s = i ) = \mathrm{P}( X_t = j \mid X_0 = i )
\end{aligned}
\end{equation}

Chapman-Kolmogoroff equation: 

\begin{equation}
\sum_{k}\mathrm{p_s}(i,k)\mathrm{p_t}(k,j) = \mathrm{p_{s+t}}(i,j)    
\end{equation}

Jump rate:

\begin{equation}
\mathrm{q}(i,j) =  \left.\frac{\mathrm{d}\mathrm{p_t}(i,j)}{\mathrm{dt}} \,\right|_{t=0} = \lim_{t \to 0^+} \frac{\mathrm{p_t}(i,j)}{t}, \text{if } i\neq j     
\end{equation}

\section{}


\chapter{State-of-the-art}

\chapter{Analysis and design}

\chapter{Realisation}

\setsecnumdepth{part}
\chapter{Conclusion}


\bibliographystyle{iso690}
\bibliography{mybibliographyfile}

\setsecnumdepth{all}
\appendix

\chapter{Acronyms}
% \printglossaries
\begin{description}
	\item[GUI] Graphical user interface
	\item[XML] Extensible markup language
\end{description}


\chapter{Contents of enclosed CD}

%change appropriately

\begin{figure}
	\dirtree{%
		.1 readme.txt\DTcomment{the file with CD contents description}.
		.1 exe\DTcomment{the directory with executables}.
		.1 src\DTcomment{the directory of source codes}.
		.2 wbdcm\DTcomment{implementation sources}.
		.2 thesis\DTcomment{the directory of \LaTeX{} source codes of the thesis}.
		.1 text\DTcomment{the thesis text directory}.
		.2 thesis.pdf\DTcomment{the thesis text in PDF format}.
		.2 thesis.ps\DTcomment{the thesis text in PS format}.
	}
\end{figure}

\end{document}
